\documentclass{article}
\usepackage[utf8]{inputenc}
\usepackage{graphicx}
\usepackage{listings}
\usepackage{xcolor}
\usepackage{hyperref}

\title{Sistem Koreksi Ejaan Bahasa Inggris dengan Pendekatan N-Gram}
\author{Penulis}
\date{\today}

\begin{document}

\maketitle

\section{Pendahuluan}
Sistem koreksi ejaan ini dibangun untuk mendeteksi dan memperbaiki kesalahan pengetikan dalam teks bahasa Inggris. Sistem ini menggunakan model statistik berbasis frekuensi kata dan konteks kalimat dengan memanfaatkan n-gram. Berikut kemampuan utama sistem:

\begin{itemize}
    \item Deteksi kata salah eja dalam kalimat/paragraf
    \item Saran koreksi berdasarkan konteks kalimat
    \item Evaluasi kinerja dengan berbagai metrik
\end{itemize}

\section{Cara Kerja}
\subsection{Model Bahasa}
Sistem menggunakan korpus \textit{big.txt} untuk membangun:
\begin{itemize}
    \item \textbf{Unigram}: Frekuensi kata tunggal
    \item \textbf{Bigram}: Pola kemunculan pasangan kata berurutan
    \item \textbf{Trigram}: Pola kemunculan triplet kata berurutan
\end{itemize}

\subsection{Model Kesalahan}
Menggunakan operasi edit dengan jarak Levenshtein:
\begin{itemize}
    \item \textbf{Edits1}: Semua kemungkinan perubahan 1 operasi edit (hapus, sisip, ganti, transpose)
    \item \textbf{Edits2}: Kombinasi 2 operasi edit
\end{itemize}

\subsection{Alur Kerja}
\begin{enumerate}
    \item Ekstraksi kata dari teks input
    \item Pemeriksaan keanggotaan dalam kamus
    \item Generasi kandidat koreksi
    \item Seleksi kandidat terbaik menggunakan probabilitas dan konteks
\end{enumerate}

\section{Implementasi}
Kode utama menggunakan Python dengan struktur kelas \texttt{SpellingCorrector}:

\lstset{
    basicstyle=\ttfamily\small,
    breaklines=true,
    frame=single,
    language=Python,
    numbers=left
}

\begin{lstlisting}[caption=Klas Utama SpellingCorrector]
class SpellingCorrector:
    def __init__(self, corpus_file='big.txt'):
        self.WORDS = Counter()
        self.BIGRAMS = defaultdict(Counter)
        self.TRIGRAMS = defaultdict(Counter)
        self.N = 0
        self.load_corpus(corpus_file)
    
    def _build_ngrams(self, all_words):
        # Implementasi pembuatan bigram dan trigram
        ...
    
    def correction(self, word, context=None):
        # Logika seleksi kandidat dengan konteks
        ...
\end{lstlisting}

\section{Evaluasi}
\subsection{Metrik Evaluasi}
Sistem dievaluasi menggunakan 4 dataset berbeda:
\begin{itemize}
    \item spell-testset1.txt: Akurasi 74.81\%
    \item spell-testset2.txt: Akurasi 67.50\% 
    \item wikipedia.txt: Akurasi 60.69\%
    \item aspell.txt: Akurasi 43.31\%
\end{itemize}

\begin{table}[h]
\centering
\caption{Perbandingan Kinerja pada Berbagai Dataset}
\begin{tabular}{|l|c|c|c|c|}
\hline
\textbf{Dataset} & \textbf{Akurasi} & \textbf{Presisi} & \textbf{Recall} & \textbf{F1} \\
\hline
spell-testset1 & 74.81\% & 77.99\% & 94.84\% & 85.59\% \\
spell-testset2 & 67.50\% & 75.63\% & 86.26\% & 80.60\% \\
wikipedia & 60.69\% & 68.86\% & 83.60\% & 75.51\% \\
aspell & 43.31\% & 54.12\% & 68.45\% & 60.45\% \\
\hline
\end{tabular}
\end{table}

\subsection{Contoh Koreksi}
Input:
\begin{verbatim}
Thiss is an exampel of a paragraf with severall mispelled words.
The spellin correcton systm should be able to detecct and sugggest...
\end{verbatim}

Output:
\begin{verbatim}
this is an example of a paragraph with several dispelled words.
The spelling correction system should be able to detect and suggest...
\end{verbatim}

\section{Kesimpulan}
Sistem ini menunjukkan:
\begin{itemize}
    \item Kemampuan deteksi kesalahan yang baik (Recall 94.84\%)
    \item Ketergantungan kuat pada kualitas dataset pelatihan
    \item Performa menurun pada kata yang tidak ada dalam korpus
    \item Potensi peningkatan dengan model konteks yang lebih kompleks
\end{itemize}

Kode lengkap tersedia di repositori GitHub: \url{https://github.com/example/spelling-corrector}

\end{document}
